%%%%%%%%%%%%%%%%%%%%%%%%%%%%%%%%%%%%%%%%%
% Iulia Pasov - CV
% using Modern CV LaTeX Template
% Version 1.1 (9/12/12)
%
% This template has been downloaded from:
% http://www.LaTeXTemplates.com
%
% Original author:
% Xavier Danaux (xdanaux@gmail.com)
%
% License:
% CC BY-NC-SA 3.0 (http://creativecommons.org/licenses/by-nc-sa/3.0/)
%
%%%%%%%%%%%%%%%%%%%%%%%%%%%%%%%%%%%%%%%%%

%------------------------------------------------------------------------------
%	PACKAGES AND OTHER DOCUMENT CONFIGURATIONS
%------------------------------------------------------------------------------

\documentclass[11pt,a4paper,sans]{moderncv}

\moderncvstyle{classic}
\moderncvcolor{grey}

\usepackage{lipsum}

\usepackage[scale=0.85]{geometry} % Reduce document margins
\setlength{\hintscolumnwidth}{3.5cm}

%------------------------------------------------------------------------------
%	NAME AND CONTACT INFORMATION SECTION
%------------------------------------------------------------------------------

\firstname{Iulia}
\familyname{Pa\c{s}ov}

\title{Curriculum Vitae}
\address{Bucharest}{Romania}
\extrainfo{(+4) 0746 183 944 \\ iulia.pasov@gmail.com}

%------------------------------------------------------------------------------

\begin{document}

\makecvtitle

%------------------------------------------------------------------------------
%	PERSONAL INFORMATION
%------------------------------------------------------------------------------
%\section{Personal Information}
%\cventry{Date of Birth}{July 19\textsuperscript{th} 1987}{}{}{}{}
%\cventry{Place of Birth}{Drobeta Turnu-Severin, Romania}{}{}{}{}
%\cventry{Nationality}{Romanian}{}{}{}{}

%------------------------------------------------------------------------------
%	WORK EXPERIENCE SECTION
%------------------------------------------------------------------------------

\section{Experience}

\cventry{December 2014-present}
{Big Data \& MAchine Learning Engineer}{\textsc{Avira Soft}}{Bucharest}{Romania}
{\textbf{Data Science Team} \\
    \textbf{Responsibilities:} \\
    - data aquisition, data preparation, data analysis \\
    - building and implementing models from data \\
    - using data models to classify data or predict behaviors \\
    \textbf{Keywords}: Machine Learning, Data Mining, Big Data, Data Science, 
    Natural Language Processing, Hadoop, Spark, Hive, Couchbase, Scala, Python }

\cventry{April 2013-December 2014}
{R\&D Software Engineer}{\textsc{Eau de Web}}{Bucharest}{Romania}
{Projects: \\ 
\textbf{ $\bullet$ European Environmental Agency (EEA) Website Services } \\ 
\textbf{Responsibilities}: \\ 
-build a new Advanced Search tool for the EEA website (http://www.eea.europa.eu), 
synchronized with a SPARQL endpoint, to support queries on linked data \\ 
-develop a river plugin for ElasticSearch to harvest data from the EEA endpoint \\
\textbf{Keywords}: Semantic Web, Linked Data, RDF, SPARQL, Virtuoso, 
ElasticSearch, Open Data, Big Data, JAVA, Jena \\
\textbf{ $\bullet$ State of the Environment Reporting Information System (SERIS) } \\
\textbf{Responsibilities}: develop a semantically relevant mapping to convert 
the SERIS database (PostgreSQL) 
to RDF using relevant W3C recommended namespaces  \\
\textbf{Keywords}: Semantic Web, RDF, PostgreSQL, Python\\ 
\textbf{ $\bullet$ Digital Agenda Scoreboard } \\ 
\textbf{Responsibilities}: configure the Elda semantic web browser for the data \\ 
\textbf{Keywords}: Semantic Web, RDF, SPARQL, Open Data, Virtuoso, Java EE }

%-----------------------------------------------
\cventry{Oct 2008--Sept2013}
{Associate Teaching Assistant}
{\textsc{POLITEHNICA University of Bucharest}}
{Bucharest}{Romania}
{ \textbf{Courses:} Algorithm Analysis (2008-2009, 2010-2013), 
Algorithm Design (2008-2009, 2010-2011, 2012-2013), 
Algorithm Design and Complexity (2012-2013), 
Computer Programming (2012-2013), 
Programming Paradigms (2010-2011)  \\ 
\textbf{Responsibilities:} \\  
- teach seminars \\ 
- evaluate students \\ 
- propose homework and exercises for exams}

%------------------------------------------------

\cventry{Jul 2012--Sept 2012}
{Summer Student}
{\textsc{CERN}}
{Geneva}{Switzerland}
{ Keywords: JavaScript, C++, ROOT, Graphics. \\ 
\textbf{Purpose:}  
- offer a portable way to monitor or inspect ROOT files on mobile devices. \\ 
\textbf{Responsibilities:} \\ 
- adapt a JavaScript library to ROOT's specification needs for 1D, 2D and 3D. \\ 
- attend Particle Physics lectures, seminars and demonstrations. }

\cventry{Jul 2011--Sep 2011}
{Intern}
{\textsc{Semsoft}}
{Rennes}{France}
{Keywords: Complex Data Reconciliation, Java,  Knowledge Representation, 
    Natural Language Processing, Normalization, RDFS, OWL, SPARQL. \\ 
\textbf{Responsibilities:} \\ 
- desig a representation for SPARQL queries (disjunctive form). \\ 
- desig algorithms to normalize existing SPARQL queries. \\ 
- create a reconciliation module for multiple databases (RDFS+) and algorithms 
for text, numbers and complex data reconciliation.}

%\cventry{Oct 2010--Jun 2011}{Academic Relations Officer}{\textsc{AI-MAS Winter 
%Olympics}}{Bucharest}{Romania}{Manage the interaction with the academic 
%partners of the event, both from Romania and abroad}

%------------------------------------------------------------------------------
%	EDUCATION SECTION
%------------------------------------------------------------------------------

\newpage
\section{Education}
\cventry{2011--2012}
{Master 2 Recherche - Extraction des Connaissances \`a partir des Donn\'ees}
{Universit\'e Lumi\`ere (Lyon II)}
{Lyon}{France}
{Thesis:\textbf{Machine Learning Techniques for Classifying Forum Messages}}

\cventry{2010--2013}
{Master of Science in Artificial Intelligence}
{POLITEHNICA University of Bucharest}
{Bucharest}{Romania}
{Thesis:\textbf{An Operational Model of Emotions for 2APL Agents}}

\cventry{2006--2010}
{Bachelor in Computer Science}
{POLITEHNICA University of Bucharest}
{Bucharest}{Romania}
{Thesis:\textbf{A Rule-based System for Evaluating Students' Participation to a Forum}}

%\cventry{2002--2010}{Undergraduate}{"Traian" National College}{Drobeta Turnu-Severin}{Romania}{}

%----------------------------------------------------------------------------------------
%	PUBLICATIONS SECTION
%----------------------------------------------------------------------------------------

\section{Publications}

\cventry{2011}
{I. Pa\c{s}ov, \c{S}t. Tr\u{a}u\c{s}an-Matu, and T. Rebedea}
{A rule-based system for evaluating students' participation to a forum}
{First International K-Teams Workshop on Semantic and Collaborative Technologies for the Web}
{Bucharest, Romania}{}
\cventry{2011}
{I. Pa\c{s}ov, \c{S}t. Tr\u{a}u\c{s}an-Matu, and T. Rebedea}
{A System for the Evaluation of the Participation of a Student to a Discussion Forum}
{Journal of Human-Computer Interaction}
{Bucharest, Romania}{}
%\cventry{}{}{} {}{}{}

%----------------------------------------------------------------------------------------
%	SKILLS AND EXPERTISE SECTION
%----------------------------------------------------------------------------------------

\section{Skills}

\cventry{}{Languages}{}{}{}{}
\cvitem{Proficient}{\textsc{Python, Scala}}
\cvitem{Prior experience}{\textsc{Java, C,  C\#, C++, JavaScript, R, Prolog, 
Scheme, RDF}}

\cventry{}{Databases \& Data Warehouses}{}{}{}{}
\cvitem{Proficient}{\textsc{Virtuoso, Hive}}
\cvitem{Prior experience}{\textsc{MySQL, MongoDB, Couchbase}}
%\cvitem{Procedural}{\textsc{C}(proficient)}
%\cvitem{Object Oriented}{\textsc{Java}(proficient)\textsc{, C++}(proficient),
%\textsc{ C\#}(prior experience)}
%\cvitem{Logic}{\textsc{Prolog}(prior experience)}
%\cvitem{Functional}{\textsc{Scheme}(proficient),\textsc{Haskell}(proficient)}
%\cvitem{Scripting}{\textsc{JavaScript}(prior experience)}

\cventry{}{Tools}{}{}{}{}
\cvitem{Proficient}{\textsc{Git, Elasticsearch, Kibana}}
\cvitem{Prior Experience}{\textsc{Subversion, Virtualenv, Jenkins, Weka, 
Elasticsearch, Jena, RDFLib, Elda, JQuery, Django, Zope}}

%\cvitem{Scientific/Statistics}{Matlab, R, Weka}
%\cvitem{Ontologies}{Prot\'eg\'e}
%\cvitem{Multi-Agent}{Jade}
%\cvitem{Graphics}{Photoshop}
%\cvitem{Animation}{Adobe Flash}

%\cventry{}{Office Automation}{}{}{}{}
%\cvitem{}{\LaTeX, Office\texttrademark,    }

%----------------------------------------------------------------------------------------
%	LANGUAGE SKILLS SECTION
%----------------------------------------------------------------------------------------

\section{Languages}

\cvitemwithcomment{Romanian}{Native proficiency }{}
\cvitemwithcomment{English}{Bilingual proficiency}{Cambridge CAE Certificate}
\cvitemwithcomment{French}{Limited working proficiency}{}
%\cvitem{Basic}{\textsc{java}, Adobe Illustrator}
%\cvitem{Intermediate}{\textsc{python}, \textsc{html}, \LaTeX, OpenOffice, Linux, Microsoft Windows}
%\cvitem{Advanced}{Computer Hardware and Support}

%----------------------------------------------------------------------------------------
%	LANGUAGES SECTION
%----------------------------------------------------------------------------------------

%\section{Languages}

%\cvitemwithcomment{English}{Mothertongue}{}
%\cvitemwithcomment{Spanish}{Intermediate}{Conversationally fluent}
%\cvitemwithcomment{Dutch}{Basic}{Basic words and phrases only}

%----------------------------------------------------------------------------------------
%	INTERESTS SECTION
%----------------------------------------------------------------------------------------

\section{Interests}

%\subsection{Scientific Interests}
\renewcommand{\listitemsymbol}{-~} % Changes the symbol used for lists

\cvlistdoubleitem{Artificial Intelligence}{Algorithms}
\cvlistdoubleitem{Affective Computing}{Machine Learning}
\cvlistdoubleitem{Game Theory}{Semantic Web}
\cvlistdoubleitem{Data Analysis}{Information Retrieval}
\cvlistitem{Natural Language Processing and Understanding}

%\subsection{Other Interests}
%\cvlistdoubleitem{Photography}{Hiking}
%\cvlistdoubleitem{Reading}{Running}

\section{Updated}
\cvitem{}{June 11$^{th}$, 2015}

%\newpage

%\section{Additional Information}
%\subsection{Personal and Faculty Projects}

%\cventry{2012}{Text Mining for Sentiment Analysis}{Faculty Project}{Developed in Java}{}{Usually sentiment analysis and opinion mining problems are solved either with natural language processing algorithms or statistical learning, but best results are always obtained when the two approaches are combined. For this project I have decided to test both approaches, individually and combined. Therefore, the application can be divided into three parts: sentiment analysis with SentiWordNet, sentiment analysis with machine learning (Na\"ive Bayes from \textsc{LingPipe}) and upgraded statistical learning (machine learning, stemming, removing stop words, grouping similar words).}

%\cventry{2012}{Ant Colony System for Simulating Human Memory}{Faculty Project}{developed in Java}{}{The purpose of this project was to simulate the human memory using an ant colony system. The main idea was to represent the strength of a memory with pheromones. The human memory is very complex and can be divided into several categories and situations. Therefore this project only considers storing information from articles read or studied over time. While the subject is reading or studying an article or practicing different applications of theories, ants leave traces in his brain's graph showing that the particular information was processed or reprocessed. Ants choose path according to probabilities. A more probable route will always be one with a strong pheromone trail. The strength of those traces is proportionally to the time spent to study the topic and vanishes in time. One advantage of this approach over the classical ones based on Ebbinghaus' forgetting curve is that whenever the subject studies new articles related to the topic, new connections are explored and the time spent to remember the notions decreases. Moreover, domains that are not searched but are highly related to them can be remembered a bit so their forgetting curves change.}

%\cventry{2011}{A Collaborative System for Medical Image Analysis and Diagnosis}{Faculty Project}{Developed in JavaScript, PHP, Flash}{}{The purpose of this project was to develop a collaborative environment for medical image analysis. The product ensured the communication of several people, each able to upload the medical examinations of an anonymous patient, to annotate them and discuss in a private chat room. The application brought advantages for both doctors (increase experience, learn new techniques) and patients (lower costs for traveling, the opportunity to be examined, in the same time, by several doctors from different locations), who became indirect users of the applications. }

%\cventry{2010}{A Rule-based System for Evaluating Students' Participation to a Forum}{Bachelor Thesis}{}{}{The purpose of this study is to provide a tool used in computer mediated communication and designed to support educational experiences. The application is based on the Community of Inquiry model, which consists of three elements: the cognitive presence, the social presence and the teaching presence. It implements a rule-based algorithm that evaluates participations to forums according to a complex set of patterns. }

%\cventry{2008}{Puzzle Works: Tablet for Education}{Microsoft Imagine Cup}{Developed in C\#, HTML}{}{ Our team tried to build a tablet-like device, easy to use, autonomous,  robust enough to resist in the harshest  conditions, and embed in it the best  tools for education. We wanted to use touch-screen technology, wireless connectivity and to create a DC system so that in the cases where there is no power grid, solar energy can be used. The software for this device should have made it more than just an augmentation to an existing education system. We wanted to go with it even to the places where there are no schools or  teachers.  The set of applications should have included writing, reading,  math, music and drawing. We have approximated the cost of one device to be less than 99\$. Unfortunately the project was rejected by Microsoft and it was not completed.}











\end{document}
